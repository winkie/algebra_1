\section{Китайская теорема об остатках. Целочисленный вариант. Использование в модулярной арифметике.}

\begin{thm}[Целочисленный вариант китайской теоремы об остатках]
  Пусть $ m_1, \dots, m_k \in \mathbb{Z} $ - попарно взаимнопростые числа. Тогда
  \[\forall x_1, \dots, x_k, ~~~ 0 \le x_i < m_i  \]
  \[\exists ! x \in \mathbb{Z}_{m}: x \equiv x_i \mod m_i, ~i = \overline{1, k}.\]
Существует изоморфизм колец:
\[\phi:~~~\mathbb{Z}\bigg / ( {\underset{i = 1}{\overset{k}\prod} m_i} ) \mathbb{Z} \rightarrow \underset{i = 1}{\overset{k}\prod} \mathbb{Z}/m_i\]

\end{thm}
\begin{proof}
  $ m =  {\underset{i = 1}{\overset{k}\prod} m_i} $ \newline
  Построим $ \phi $ - изоморфизм.
  \[ \phi(x) = (x \; {\rm mod} \; m_1, x \; {\rm mod} \; m_2, \dots, x \; {\rm mod} \; m_k) \]
  $ \phi $ - инъективно. Действительно, пусть $ \phi $ - не инъективно, тогда
  \[ \exists x, y \in \mathbb{Z}/m\mathbb{Z} : ~ \forall i ~~ x - y ~ \vdots ~ m_i \overset{(m_{i}, m_{j}) = 1}{\Rightarrow} x - y ~ \vdots ~ m \]
  Но $ \forall x, y \in \mathbb{Z}/m\mathbb{Z} ~ |x - y| < m $, а значит $ x = y $. \newline
  $ \phi $ - сюръективно. Действительно, заметим, что $ \phi $ действует между равномощными множествами. И далее по принципу Дирихле: любое инъективное
  отображение между двумя равномощными множествами сюръективно.
\end{proof}

С помощью алгоритма Евклида и целочисленного варианта китайской теоремы об остатках может быть построен
конструктивный алгоритм восстановления числа по его остаткам (от деления на взаимнопростые числа). \newline
Например в алгоритме $ RSA $ вычисления ведутся по модулю $ n = pq $, где $ p, q $ - большие простые числа,
что делает эти вычисления в $ \mathbb{Z}/n\mathbb{Z} $ достаточно долгими. Китайская теорема об остатках 
позволяет вести эти вычисления в $ \mathbb{Z}/p\mathbb{Z} \times \mathbb{Z}/q\mathbb{Z} $. \newline

(Я не уверен, что всё про применение КТО в модулярной арифметике..)

