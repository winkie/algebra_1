\section{Теорема Берлекэмпа}

Вообще, идея Берлекэмпа состоит в применении КТО. Если $(c_1, c_2, ..., c_m)$ является произвольным набором чисел из $F_p$, то из КТО вытекает, что $\exists!~g(x) \in k[x]$
такой, что 
\[
\begin{array}{l}
g(x) \equiv c_1 \mod f_1 \\
g(x) \equiv c_2 \mod f_2 \\
g(x) \equiv c_3 \mod f_3 \\
\dots \\
g(x) \equiv c_m \mod f_m \\
\end{array}
\]

Полином $g(x)$ предоставляет способ получения множителей $f(x)$, так как при $m \geqslant 2$ и $c_1 \neq c_2$ мы получим
$\text{НОД}(f(x), g(x) - c_1)$, делящийся на $f_1(x)$, но не на $f_2(x)$.

\begin{lem}
$F_p \subset k, char(k) = p$ \\
$\forall x \in k~~~x \in F_p \Leftrightarrow x^p = x $
\end{lem}
\begin{proof}
  $ [\Rightarrow] $ Непосредственно следует из малой теоремы Ферма. \newline
  $ [\Leftarrow] $ $x^p = x$ - имеет $p$ корней. Возьмем $\alpha_1 \in F_p$ - очевидно, корень. Так мы можем предъявить
все $p$ корней, и других быть не может.
\end{proof}

\begin{thm} [Берлекэмпа]
\[
\begin{array}{l}
~~~~~g^p - g~\vdots~f \\
(g^p \equiv g \mod f)
\end{array} \Leftrightarrow \begin{array}{l}
                            \exists c_1, c_2, ..., c_m \in F_p: \\
                            ~~~~~g - c_i~\vdots~f_i
                            \end{array}
\]
\end{thm}
\begin{proof}
$[\Leftarrow]$
\[
\begin{array}{l}
g - c_1 \vdots f_1 \Leftrightarrow g - c_1 \equiv 0 \mod f_1 \Leftrightarrow g \equiv c_1 \mod f_1 \\
g^p \equiv c_1^p \equiv [\text{малая т. Ферма}] \equiv c_1 \equiv g \mod f_1
\end{array}
\]
Таким образом, 
\[
\begin{array}{l}
g^p - g ~ \vdots ~ f_1 \\
g^p - g ~ \vdots ~ f_2 \\
g^p - g ~ \vdots ~ f_3 \\
\vdots \\
g^p - g ~ \vdots ~ f_m \\
\end{array}
\Leftrightarrow g^p - g ~ \vdots ~ f
\]
$[\Rightarrow]$
\[
\begin{array}{l}
g^p - g ~ \vdots f \Rightarrow g^p - g ~ \vdots ~ f_i \Rightarrow g^p - g \equiv 0 \mod f_i \Rightarrow \\
g^p \equiv g \mod f_i \buildrel{\text{по лемме}}\over{\Rightarrow} g \equiv c_i \mod f_i,~~~c_i \in F_p
\end{array}
\]
\end{proof}

Пусть теперь мы нашли $g$, такой, что $g^p \equiv g \mod f$. По теореме, $\exists c \in F_p:~g - c ~ \vdots ~ f_1$.
Ищем
\begin{align}
\text{НОД}(g - 0&, f) \notag \\
\text{НОД}(g - 1&, f) \notag \\
\text{НОД}(g - 2&, f) \notag \\
&\vdots \notag \\
\text{НОД}(g - (p - 1)&, f), \notag
\end{align}
и один из делителей будет нетривиальным.



