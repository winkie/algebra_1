\section{Неприводимые полиномы над полем. Неразложимые элементы кольца. Понятие факториального кольца. Существование
неприводимых полиномов над конечными полями.}

\begin{defn}
Многочлен $f \in k[x]$ называется неприводимым над полем $k$, если он имеет положительную степень и равенство
$f = gh, g \in k[x], h \in k[x]$ может выполняться только в том случае, когда либо $g$, либо $h$ является постоянным
многочленом.
\end{defn}

\begin{defn}
$A$ - кольцо. \\
$a \in A$ - неразложим, если $a = g_1g_2 \Rightarrow \exists g_1^{-1} \vee \exists g_2^{-1}$
%(т.е. $g_1 \vee g_2$ - единица кольца)
\end{defn}

\begin{defn}
Кольцо $A$ называется факториальным, если для любого элемента существует разложение на неразложимые элементы, и оно
единственно с точностью до порядка следования неразложимых сомножителей и единиц кольца.
\end{defn}

Про существование я ниче не нашел \%)