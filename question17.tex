\section{Общий вариант китайской теоремы об остатках. Применение ее к кольцу полиномов.}

\begin{defn}
Идеалы $\Id{a}, \Id{b}$ взаимнопросты, т.е. $(\Id{a}, \Id{b}) = 1$, если $\Id{a} + \Id{b} = A$
\end{defn}

\begin{thm}[Китайская теорема об остатках]
Пусть $A$ - кольцо и $\Id{a}_1, \dots, \Id{a}_k$ - попарно взаимнопростые идеалы. Тогда
\[\forall x_1, \dots, x_k, ~~~x_i \in A\]
\[\exists x \in A: x \equiv x_i \mod \Id{a}_i, ~i = \overline{1, k}.\]
Существует изоморфизм колец:
\[\phi:~~~A\bigg/\underset{i = 1}{\overset{k}\bigcap} \Id{a}_i \rightarrow \underset{i = 1}{\overset{k}\prod} A/\Id{a}_i\]
\end{thm}
\begin{proof}
Доказательство по ММИ.\\
База: $k = 2$
\[\Id{a}_1 + \Id{a}_2 = A \Rightarrow \exists y_1 \in \Id{a}_1, y_2 \in \Id{a}_2:~~~y_1 + y_2 = 1.\]
Надо показать, что
\[\exists x \in A: \begin{array}{l l}
                   x & \equiv x_1 \mod \Id{a}_1 \\
                   x & \equiv x_2 \mod \Id{a}_2 \\
                   \end{array}
\]
Предъявим $x = x_2y_1 + x_1y_2$. Покажем, что это верно:
\begin{gather}
x - x_1 = x_2y_1 + x_1y_2 - x_1 = x_2y_1 + x_1(y_2 - 1) = x_2y_1 - x_1y_1 \in \Id{a}_1 \notag \\
x - x_2 = x_2y_1 + x_1y_2 - x_2 = x_1y_2 + x_2(y_1 - 1) = x_1y_2 - x_2y_2 \in \Id{a}_2 \notag
\end{gather}
Переход:\\
Покажем, что $\Id{a}_1$ взаимнопрост с $\underset{i = 2}{\overset{k}\prod}\Id{a}_i$. Т.к. $\Id{a}_1$ взаимнопрост с
$\Id{a}_i,~i=\overline{2,k}$:
\begin{gather}
a_1 + b_1 = 1, ~~~a_1 \in \Id{a}_1, b_1 \in \Id{a}_2 \notag \\
a_2 + b_2 = 1, ~~~a_2 \in \Id{a}_1, b_2 \in \Id{a}_3 \notag \\
\dots \notag \\
a_{k-1} + b_{k-1} = 1, ~~~a_{k-1} \in \Id{a}_1, b_{k-1} \in \Id{a}_k \notag
\end{gather}
Рассмотрим произведение:
\[(a_1 + b_1)(a_2 + b_2)\cdots(a_{k-1} + b_{k-1}) = 1 \Leftrightarrow\] 
\[(\dots)a + b_1b_2 \cdots b_{k-1} = 1,\]
где $a \in \Id{a}_1$, а $b_1b_2 \cdots b_{k-1} \in \underset{i = 2}{\overset{k}\prod}\Id{a}_i$.
Аналогично получаем, что $\Id{a}_l$ взаимнопрост с $\underset{i = 1, i \neq l}{\overset{k}\prod}\Id{a}_i$.\\
Вернемся к доказательству теоремы. Пусть она верна для семейства из $k - 1$ идеалов.
Рассмотрим $\Id{a}_1$ и $\Id{b} = \underset{i = 2}{\overset{k}\prod}\Id{a}_i$. По КТО:
\[
\exists y_1: \begin{array}{l l}
              y_1 & \equiv 1 \mod \Id{a}_1 \\
              y_1 & \equiv 0 \mod \Id{b} \\
             \end{array}
\]

Аналогичным образом найдем $y_2, y_3, \dots, y_l$. Получаем:
\[
\left\{ 
\begin{array}{l l}
  y_i & \equiv 1 \mod \Id{a}_i \\
  y_i & \in \Id{a}_j, ~i \neq j\\
\end{array} \right.
\]
Предъявим $x = \underset{i = 1}{\overset{k}\sum}x_iy_i$. Действительно,
\[x - x_i = \underset{l \neq i}\sum x_ly_l + (x_iy_i - x_i) \in \Id{a}_i,\]
т.к. $\underset{l \neq i}\sum x_ly_l \in \Id{a_i}$ и $x_i(y_i - 1) \in \Id{a}_i$.

Таким образом имеем сюръективное отображение $f: A \rightarrow \underset{i = 1}{\overset{k}\prod} A/\Id{a}_i$. \\
Заметим, что ядро данного гомоморфизма есть $\underset{i = 1}{\overset{k}\bigcap}\Id{a}_i$. По теореме о гомоморфизме
имеем изоморфизм:
\[A\bigg/\underset{i = 1}{\overset{k}\bigcap \Id{a}_i} \cong \underset{i = 1}{\overset{k}\prod} A/\Id{a}_i\]
\end{proof}

Рассмотрим применение теоремы к кольцу полиномов.

\begin{thm}
$K[x]$ - кольцо полиномов над полем $K$.\\
$f_1, \dots, f_k$ - полиномы такие, что $(f_i) + (f_j) = K[x] = (1), i \neq j$.
Тогда для любого набора остатков $\forall r_1, \dots, r_k, ~r_i \in K[x]$:
\[\exists f \in K[x]:~~~(f - r_i)~\vdots~f_i, ~i = \overline{1, k}.\]
\end{thm}

