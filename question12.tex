\section{Поля и области целостности. Характеризация простого и максимального идеалов в терминах факторкольца.}

\begin{defn}
Кольцо называтся областью целостности, если в нем нет делителей нуля:
\[\forall x \neq 0, y \neq 0 \Rightarrow xy \neq 0\]
\end{defn}

Пример: $Z$ - область целостности.

\begin{defn}
Поле - кольцо, в котором:
\[\forall x \in A, x \neq 0~~~\exists x^{-1}: xx^{-1} = 1\]
\end{defn}

\begin{thm}
Идеал $\Id{P}$ прост тогда и только тогда, когда $A/\Id{P}$ - область целостности
\end{thm}
\begin{proof}
$[\Rightarrow]$ $\Id{P}$ прост.
\[\overline{0} \neq \overline{x} \in A/\Id{P}\]
\[\overline{0} \neq \overline{y} \in A/\Id{P}\]
Покажем, что $\overline{x} \cdot \overline{y} \neq \overline{0}$.
\[x \notin \Id{P}, y \notin \Id{P} \Rightarrow xy \notin \Id{P} \Leftrightarrow \overline{xy} \neq \overline{0}
\Leftrightarrow \overline{x}\cdot\overline{y} \neq \overline{0}\]
$[\Leftarrow]$ $A/\Id{P}$ - область целостности. Пусть $\Id{P}$ - не прост. Тогда $\exists a_1, a_2: a_1a_2 \in \Id{P}$,
но $a_1 \notin \Id{P}~\&~a_2 \notin \Id{P}$. Рассмотрим соответствующие классы эквивалентности:
\[\overline{a_1} \neq \overline{0}~\&~\overline{a_2} \neq \overline{0} \]
\[\overline{a_1a_2} = \overline{a_1}\cdot\overline{a_2} = \overline{0}\]
Но $A/\Id{P}$ - область целостности. Получили противоречие.
\end{proof}

ЗАМЕЧАНИЕ: $1 \in \Id{B} \Leftrightarrow \Id{B} = A$.

\begin{defn}
$S \subset A$, тогда $(S)$ - идеал, порожденный множеством $S$, т.е. пересечение всех идеалов, содержащих $S$.
\end{defn}


\begin{thm}
$\Id{M}$ - максимальный $\Leftrightarrow$ $A/\Id{M}$ - поле
\end{thm}
\begin{proof}
$[\Rightarrow]$ $\Id{M}$ - максимальный идеал. Покажем, что $A/\Id{M}$ - поле. Возьмем ненулевой элемент и найдем обратный
к нему.
\[\overline{x} \in A/\Id{M},~~~\overline{x} \neq \overline{0} \Leftrightarrow x \notin \Id{M}\]
\[(\Id{M}\bigcup\{x\}) = A \Rightarrow 1 \in (\Id{M}\bigcup\{x\})\]
\[(\Id{M}\bigcup\{x\}) = xA + \Id{M}\]
\[\exists y \in A, ~ m_1 \in \Id{M}: 1 = xy + m_1 \Leftrightarrow \overline{1} = \overline{x}\cdot\overline{y} + \overline{0}
 = \overline{xy}
\]
$[\Leftarrow]$ $A/\Id{M}$ - поле. Пусть $\Id{M}$ не максимальный. Тогда $\exists \Id{M}_1: \Id{M} \subset \Id{M}_1
\subset A$. Возьмем $x \in \Id{M}_1\setminus\Id{M}$:
\[(\Id{M}\bigcup\{x\}) \subset \Id{M_1} \Rightarrow 1 \notin (\Id{M}\bigcup\{x\})\]
Рассмотрим $\overline{x} \in A/\Id{M}$. Т.к. $A/\Id{M}$ - поле, то $\exists \overline{y} \in A/\Id{M}:~
\overline{x}\cdot\overline{y} = \overline{1}.$
\[\exists m_1, m_2 \in \Id{M}: (x + m_1)(y + m_2) = 1 = xy + m_1y + m_2x + m_1m_2 = 1 \Rightarrow 1 \in
(\Id{M}\bigcup\{x\})\]
Получили противоречие.
\end{proof}

\begin{thm}
Максимальный идеал простой.
\end{thm}
\begin{proof}
Пусть $\Id{M}$ - максимальный идеал, и пусть $x, y \in A$ таковы, что $xy \in \Id{M}$. Предположим, что $x \in \Id{M}$.
Тогда $\Id{M} + Ax$ - идеал, строго содержащий $\Id{M}$ и,  стало быть, равный $A$. Следовательно, мы можем написать
\[1 = u + ax,\] где $u \in \Id{M}$ и $a \in A$. Умножая на $y$, получаем $y = yu + axy$, откуда $y \in \Id{M}$ и $\Id{M}$,
таким образом, простой. 
\end{proof}