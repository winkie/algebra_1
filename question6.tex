\section{Действие группы на множестве. Орбиты. Разбиение множества на орбиты и формула орбит. Стабилизатор.}

\begin{defn}
Под действием группы $G$ на множестве $X$ понимается: $s: G \times X \rightarrow X$ со свойствами:
\begin{enumerate}
  \item $s(g_1, s(g_2, x)) = s(g_1 g_2, x)$
  \item $s(e, x)) = x$
\end{enumerate}
\end{defn}

\vspace{0.5cm}
$x \mapsto s(g, x)$ обозначим как $i_g$. Обратное действие - $s^{-1}(g, x) = s(g^{-1}, x)$. \\
\par Например, $s(g, x) = g \cdot x$ (Что-то я не понял. Дальше мы только это действие и используем..):
\begin{enumerate}
 \item $g_1(g_2x) = (g_1g_2)x$
 \item $ex = x$
\end{enumerate}

\begin{defn}
Орбитой точки $x \in X$ назовем множество $G_x = \{ s(g, x) | g \in G \}$
\end{defn}

\begin{lem}
Множество орбит - разбиение множества $X$. Орбиты либо совпадают, либо не пересекаются.
\end{lem}
\begin{proof}
Пусть $y \in G_{x_1} \bigcap G_{x_2}$. Это значит, что $ \exists g_1, g_2: y = g_1x_1 = g_2x_2$.
Рассмотрим элемент $\widetilde{y} = gx_1$ из орбиты $G_{x_1}$. Но $\widetilde{y} = gg^{-1}_1y = gg^{-1}_1g_2x_2$. 
Значит $\widetilde{y}$ из орбиты $G_{x_2}$. Следовательно орбиты совпадают.
\end{proof}

\begin{defn}
Назовем стабилизатором точки $x \in X$ множество $S_x \in G: S_x = \{g \in G | gx = x \}$.
\end{defn}

\begin{lem}
Стабилизаторы различных точек сопряжены в одной. $x$ и $x_1$ - точки одной орбиты,
тогда $\exists g: S_x = gS_{x_1}g^{-1}$.
\end{lem}
\begin{proof}
$x_1 = gx$ - т.к. они с одной орбиты. $x = g^{-1}x_1$. Рассмотрим $w \in S_x: wx = x$. 
\begin{center}
$wg^{-1}x_1 = g^{-1}x_1$\\
$gwg^{-1}x_1 = x_1$\\
$gwg^{-1} \in S_{x_1}$
\end{center}
\end{proof}

Орбиты будем обозначать $O_x = G_x$.

\begin{thm}
$|O_x| = \sharp(O_x) = [G : S_x], \forall x \in X$
\end{thm}
\begin{proof}
$G \rightarrow O_x$, $x \mapsto gx$ (не в теме :|)\\
Введем эквивалентность: $g_1 ~ g_2 {\buildrel{def}\over{\Leftrightarrow}} g_1x = g_2x
\Leftrightarrow g^{-1}_1g_2x = x \Leftrightarrow g^{-1}_1g_2 \in S_x$.
\end{proof}

\begin{thm}
Формула орбит. \\
$X = {\underset{x \in Orb(X)}{\bigcup}}O_x$, ($x \in Orb(x)$ - берем по одному представителю со всех орбит) \\
$|X| = {\underset{x \in Orb(X)}{\sum}}|O_x| = {\underset{x \in Orb(X)}{\sum}}[G : S_x]$
\end{thm}


