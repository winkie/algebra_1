\section{Существование максимального идеала в кольце. Лемма Цорна.}

\begin{defn}
  Частично упорядоченное множество $ Y $ называют \emph{цепью} или \emph{линейно упорядоченным множеством}, если
  $ \forall x, y \in Y \; x < y $ или $ x > y $.
\end{defn}

\begin{defn}
  Пусть $ Y $ - цепь, $ A \subset Y $. 
  Тогда $ y \in Y $ называют \emph{верхней гранью} для $ A $, если $ \forall a \in A \; y > a $
\end{defn}

\begin{defn}
  Пусть $ Y $ - цепь. Тогда $ m \in Y $ называют \emph{максимальным элементом}, если $ \forall y \in Y \; m > y $
\end{defn}

\begin{lem}[Цорна]
  Частично упорядоченное множество, в котором любая цепь имеет верхнюю грань, содержит максимальный элемент.
\end{lem}

\begin{thm}
  В любом кольце существует максимальный идеал. \newline
  $ \forall A $ - кольца. $ \exists \Id{m} \subset A :$ $ \Id{m} $ - максимальный идеал.
\end{thm}
\begin{proof}
  Пусть $ X $ - упорядоченное по включению множество собственных идеалов в $ A $ и 
  пусть $ S \subset X $ - цепь идеалов в $ A $. Тогда рассмотрим 
  $ \Id{B} = \underset{\Id{m} \subset S}{\cup} \Id{m} $. Покажем, что $ \Id{B} $ - идеал.
  \[ x, y \in \Id{B} \Rightarrow \begin{array}{l l}
                       x \in \Id{a}_{1} \\
		       y \in \Id{a}_{2}
		     \end{array}  
		     \Id{a}_{1} \subset \Id{a}_{2} \Rightarrow x + y \in \Id{a}_{2} \subset \Id{B} \]
		     \[ x \in \Id{B} \Rightarrow x \in \Id{a}_{1} \Rightarrow \forall c \in A \; cx \in \Id{a}_{1} \subset \Id{B} \]
  К тому же $ 1 \notin \forall \Id{a} \in S \Rightarrow 1 \notin \Id{B} \Rightarrow \Id{B} \ne A $. Таким образом
  $ \Id{B} $ - верхняя грань для $ S $. Итак для произвольной цепи нашлась верхняя грань, далее по лемме Цорна $ X $ содержит 
  максимальный элемент, то есть в $ A $ существует максимальный идеал.
\end{proof}

\pagebreak

Эквивалентные утверждения:

\begin{enumerate}
  \item Любое множество может быть вполне упорядочено.
  \item Произвольное декартово произведения семейства непустых множеств непусто. 
  \item \emph{Аксиома выбора.} Для любого семейства непустых непересекающихся множеств существует множество, которое имеет
        только один элемент в пересечении с каждым множеством семейства
  \item В каждом кольце существует максимальный идеал.
  \item Лемма Цорна.
\end{enumerate}

\begin{thm}
  5 $ \Rightarrow $ 1
\end{thm}

\begin{proof}
  Пусть $ X $ - множество. Рассмотрим $ Y \subset X $ - цепь (вот почему цепь найдется в любом множестве?..) в $ X $ 
  и будем обозначать $ (Y, <) $. Введем отношение порядка
  на множестве $ \{ (Y_{s}, <_{s}) \}_s $ : $ (Y_{1}, <_{1}) < (Y_{2}, <_{2}) $, если $ Y_{1} \subset Y_{2} $, а $ <_{1} $ индуцирован $ <_{2} $.
  Пусть тогда $ S $ - цепь пар $ (Y_{s}, <_{s}) $. Рассмотрим $ Y = \cup{Y_{s}} $. $ Y $ - верхняя грань для $ S $. То есть любая цепь
  в множестве пар имеет верхнюю грань, тогда по лемме Цорна существует максимальный элемент $ (M, <) $. Предположим, что $ M \ne X $, то есть
  $ \exists x \notin M $. Рассмотрим $ M_{1} = M \cup {x} $, но тогда $ (M, <) < (M_{1}, <_{1}) $, а $ (M, <) $ - максимальный. Получили
  противоречие.
\end{proof}

Приведем еще теорему об идеалах в кольце целых чисел.

\begin{thm}
  Пусть $ p \in \mathbb{P} $. Тогда $ p\mathbb{Z} $ - максимальный идеал. 
\end{thm}

\begin{proof}
  Пусть $ (p) \subset (n) $. Тогда $ \exists m \in \mathbb{Z} ~ : ~ p = nm $, откуда
  $ n = 1 $ или $ n = p $, что и доказывает максимальность идеала, а значит и его простоту.
\end{proof}

