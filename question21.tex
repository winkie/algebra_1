\section{Алгебраическое замыкание поля. Поле разложения многочлена. Существование поля разложения. Поле Галуа
как поле разложения полинома $x^q-x$.}

\begin{thm}
Пусть $k$ - поле и $f$ - многочлен из $k[x]$ степени $\geqslant 1$. Тогда существует расширение $k \subset \BB{K}$, в котором
$f$ имеет корень.
\end{thm}
\begin{proof}
Пусть $f = gf_1$, где $g$ - неприводимый сомножитель над $k$. Тогда $(f) \subset (g)$ и $(g)$ - максимальный идеал.\\
Рассмотрим гомоморфизмы (вложения):
\[k \rightarrow k[x] \rightarrow k[x]/(g) \]
Обозначим $\BB{K} = k[x]/(g) \supset k$. \\
Возьмем моном $x \in k[x]$ и рассмотрим элемент $\overline{x} \in K$.
Покажем, что $g$ переводит класс $\overline{x}$ в $\overline{0}$, т.е. является корнем $g$.
\[\overline{x} = x + \phi \cdot g,~\text{где}~~\phi \in k[x]\]
Пусть $g(x) = \underset{i = 0}{\overset{n}\sum}a_ix^i$. Тогда
\[g(\overline{x}) = \underset{i = 0}{\overset{n}\sum}a_i(\phi g + x)^i = [\text{раскрыв скобки по биному}] = (...)g + \underset{i = 0}{\overset{n}\sum}a_i x^i\]
Таким образом, $g(\overline{x}) \equiv 0 \mod g(x)$, т.е. $g(\overline{x}) = \overline{0}$
\end{proof}

Пусть $k$ - поле, $f$ - многочлен из $k[x]$ степени $\geqslant 1$. Под \emph{полем разложения} $\BB{K}$ многочлена $f$ мы
будем понимать расширение $k \subset \BB{K}$, в котором $f$ разлагается на линейные множители, т.е.
\[f(x) = c(x-a_1)(x-a_2)\cdots(x-a_n)\]
где $a_i \in \BB{K}, i = \overline{1,n}$
%, причем $\BB{K} = k(a_1, a_2, \dots, a_n)$ порождается всеми корнями $f$.

\begin{thm}
$k$ - поле. $f$ - произвольный полином из $k[x]$.
Существует $\BB{K}_f:~~~k \subset \BB{K}_f$ и $\BB{K}_f$ - поле разложения многочлена $f$.
\end{thm}
\begin{proof}
Разложим $f$ на непривод. сомножители над полем $k$:
\[f(x) = g_1(x)g_2(x) \cdots g_k(x)\]
Построим поле $\BB{K}_1:~~~k \subset \BB{K}_1 = k[x] / (g_1)$.\\
В поле $\BB{K}_1$  $f = (x - \alpha_1) \cdot f_1$, где $\alpha_1$ - корень $g_1$. \\
Аналогично строим $\BB{K}_2, \BB{K}_3, \dots, \BB{K}_n$.
\[f(x) = (x - \alpha_1)(x - \alpha_2) \cdots (x - \alpha_n)~~~\text{в $\BB{K}_n$}\]
Поле, порожденное всеми корнями $f$: $\BB{K}_f = k(\alpha_1, \alpha_2, \dots, \alpha_n) \subset \BB{K}_n$
\end{proof}

\begin{defn}
$k \subset \BB{K}$ - алгебраическое расширение. \\
$\BB{K}$ - алгебраически замкнуто, если $\forall f \in K[x]$ имеет корень в $\BB{K}$.
\end{defn}

\begin{thm}
$k$ - поле.\\
$\exists \BB{K}:~~~k \in \BB{K}$, $\BB{K}$ - алгебраично над $k$ и алгебраически замкнуто.\\
$\BB{K}$ единственно с точностью до изоморфизма (без док-ва).
\end{thm}
\begin{proof}
Сначала построим расширение поля $k$, в котором каждый многочлен степени $\geqslant 1$ имеет корень.\\
Будем рассматривать неприводимые многочлены $f$ над полем $k$ от своей собственной переменной $x_f$.\\
Введем большое кольцо многочленов от многих переменных $k[x_1, x_2, x_3, \dots]$.\\
(Для удобства будем вместо $x_{f_i}$ писать $x_i$)\\
Рассмотрим идеал $\Id{a} = (f(x_f) \text{по всем неприводимым}~f)$. Этот идеал содержится в каком-то большем идеале $\Id{M}:
~\Id{a} \subset \Id{M}$. Покажем, что $\Id{M} \neq k[x_1, x_2, x_3, \dots]$, то есть, что $1 \notin \Id{M}$. \\
Допустим, что $\Id{M}$ содержит $1$. Тогда:
\[
\underset{i = 1}{\overset{N}\sum} g_i(x_1, x_2, x_3, \dots, x_{M_i}) \cdot f_i(x_i) = 1~~~(\ast)\\
\]
Пусть $\BB{F}$ - конечное расширение, в котором все $f_i$ имеют корень:
\[
\begin{array}{l}
f_1~\text{имеет корень}~\alpha_1 \\
f_2~\text{имеет корень}~\alpha_2 \\
\vdots \\
f_N~\text{имеет корень}~\alpha_N
\end{array}
\alpha_i \in \BB{F} \\
\]
Подставив $\alpha_i$ в ур-ние $(\ast)$ получаем:
\[ \underset{i = 1}{\overset{N}\sum} g_i(\dots) \cdot f_i(\alpha_i) = 0 = 1 !!!\]
Пусть $\Id{M}$ оказался максимальным идеалом, содержащий идеал, порожденный всеми многочленами $f(x_f)$ в $k[x_1,
x_2, x_3, \dots]$. Тогда $k[x_1, x_2, x_3, \dots] / \Id{M}$ - поле. Обозначим его через $\BB{K}_1$. Имеем каноническое вложение
$k \subset \BB{K}_1$. Все полиномы из $k[x]$ имеют корни в $\BB{K}_1$. \\
По индукции строим последовательность $k \subset \BB{K}_1 \subset \BB{K}_2 \dots$. Пусть $\BB{K} = (\underset{i}\bigcup
\BB{K}_i) \bigcup k$ - объединение полей. Очевидно, $\BB{K}$ - поле. И если $f$ - полином над $\BB{K}_n$, то его корни
лежат в $\BB{K}_{n + 1}$.
\end{proof}


Про поле Галуа см. пред. вопрос.