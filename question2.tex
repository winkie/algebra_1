\section{Вопрос 2}

Мономорфизмы, эпиморфизмы и изоморфизмы. Понятие нормального делителя (нормальной подгруппы). Факторгруппа.

\begin{defn}
  Сюръективный гомоморфизм - эпиморфизм. \newline
  Инъективный гомоморфизм - мономорфизм. \newline
  Биективный гомоморфизм - изоморфизм. \newline
  Изоморфизм $ f: G \rightarrow G $ - автоморфизм. \newline
\end{defn}

Пусть $ H \subset G $. Введем отношение эквивалентности $ \sim $ соответствующее подгруппе. $ g_1, g_2 \in G $. $ g_1 \sim g_2 $, если $ g_{1}g_{2}^{-1} \in H $

\begin{defn}
  $  \overset{\sim}{g} = \{ k \in G | k \sim g\} $ - класс эквивалентности элемента $g$ 
\end{defn}

\begin{defn}
  $ G/H $ - факторгруппа, левые смежные классы. $ \overset{\sim}{g} = Hg $
\end{defn}

Заметим, что в случае некоммутативной группы можно ввести правые смежные классы $ gH $.

\begin{thm}
  Если $ gH = Hg $, то $ G/H $ - группа.
\end{thm}

\begin{proof}  
  Введем умножение: $ \forall g_{1}H, g_{2}H \in G/H \; (g_{1}H)(g_{2}H) = g_{1}g_{2}H $.
  Проверим корректность умножения: пусть $ g_1' \sim g_1, g_2' \sim g_2 $. Тогда
  $ g_1' = g_{1}h_1, g_2' = g_{2}h_2 $, а значит $ g_{1}'g_{2}'= g_{1}h_{1}g_{2}h_{2} = g_{1}g_{2}h_{1}h_{2} $. 
  То есть $ g_{1}'g_{2}'H = g_{1}g_{2}H $. \newline
  Теперь проверим свойства умножения: 
  \begin{enumerate}
    \item $ eHgH = gH $
    \item $ g_{1}Hg_{2}Hg_{3}H = g_{1}g_{2}g_{3}H $
    \item $ gHg^{-1}H = eH $
  \end{enumerate}
\end{proof}

\begin{defn}
  $H \subset G$ назовем нормальной подгруппой, если $ \forall g \in G \; gH = Hg $ или $ gHg^{-1} = H $ или $ ghg^{-1} \in H $  
  \newline Обозначение: $H \triangleleft G$
\end{defn}

\begin{thm}
  $ G $ - абелева группа, тогда $ \forall H \subset G $ - нормальная.
\end{thm}

\pagebreak

\begin{thm}
  Ядра гомоморфизмов и только они суть нормальные подгруппы.
\end{thm}
\begin{proof}
  Сперва докажем, что если $ f: G \rightarrow W $ - гомоморфизм, то $ ker f \triangleleft G $. 
  $ g \in G, h \in ker f $, тогда $ f(ghg^{-1}) = f(g)f(h)f(g^{-1}) = f(g)f(g)^{-1} = e_W $. \newline
  Теперь покажем, что $ \forall H \triangleleft G \; \exists f $ - гомоморфизм и $ ker f = H $.
  Введем $ \pi_{H} : G \rightarrow G/H $ - канонической гомоморфизм. Пусть $ g \in G, h \in H $ 
  тогда $\pi_{H}(g) = gH, \pi_{H}(h) = hH = H $. Следовательно $ ker \pi_{H} = H $.
\end{proof}





