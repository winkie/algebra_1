\section{Группа подстановок (симметрическая группа). Четные и нечетные подстановки. 
Теорема о том, что всякая группа есть подгруппа симметричской группы (для конечных групп).}

\begin{defn} 
  \emph{Симметрической группой} $ S_{X} $ множества $ X $ называется группа автоморфизмов $ X \rightarrow X $
  относительно операции композиции и нейтрального элемента $ id_{X} : \forall x \in X, id_{X}(x) = x $. \newline
  Если $ X = \{1, 2, \cdots, n \} $, то симметричскую группу называют группой подстановок и обозначают $ S_{n} $. 
\end{defn}

Группа подстановок $ S_{n} $ допускает следующее копредставление: \newline

Образующие: \newline
$ \sigma_{1}, \sigma_{2}, \cdots ,\sigma_{n-1} $ \newline

Соотношения: \newline
$ \sigma_{i}^{2} = 1 $ \newline
$ \sigma_{i}\sigma_{j} = \sigma_{j}\sigma_{i} $, если $ |i - j| > 1 $ \newline
$ \sigma_{i}\sigma_{i+1}\sigma_{i} = \sigma_{i+1}\sigma_{i}\sigma_{i+1} $ \newline

Вообще, образующие в указанном копредставлении являются \emph{транспозициями}, то есть это такие подстановки,
которые меняют два соседних элемента местами, а остальные элементы оставляют на месте.

\begin{defn}
  Подстановка называется \emph{четной}, если она представляется в виде произведения четного числа транспозиций и 
  \emph{нечетной} в противном случае.
\end{defn}

\begin{thm}
  Любая группа - подгруппа симметрической группы.
\end{thm}
\begin{proof}
  Необходимо сопоставить каждому элементу $ g \in G $ некоторую биекцию $ G \rightarrow G $, тем самым получив вложение $ G \subset S_{G} $.
  Рассмотрим $ i_{g} : G \rightarrow G, \forall s \in G \; i_{G}(s) = gs $. 
  Осталось проверить свойства: $ i_{a} \circ i_{b} = a(bs) = (ab)s = i_{ab}, \; i_{g} \circ i_{g^{-1}} = g(g^{-1}s) = es = i_{e} $.
\end{proof}

