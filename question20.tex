\section{Характеристика поля. Простое подполе. Поля конечной характеристики. Конечные поля.
Построение полей Галуа $F_{p^n}$.}

Пусть $\mathbb{K}$ - поле.
Будем полагать $\underbrace{x + x + x + \dots + x}_{n} = nx$

Введем понятие характеристики поля:
\[char(\mathbb{K}) = p = \underset{k}{\min}\{k \cdot e = 0\}.\]
$p$ либо $\in \mathbb{N}$, либо $= 0$, если такого $k$ не существует.

Пример: $Z/pZ = F_p$ - поле, т.к. идеал pZ максимальный. Характеристика $F_p$ равна $p$.

\begin{defn}
Конечное поле - поле, состоящее из конечного числа элементов.
\end{defn}

\begin{thm}[Свойство]
Любое конечное поле имеет ненулевую характеристику.
\end{thm}

\begin{thm}[Свойство]
Характеристика поля - простое число.
\end{thm}
\begin{proof}
$] n = pq$ \\
$n \cdot e = 0 \Rightarrow (p \cdot q) \cdot e = 0 \Rightarrow p(q \cdot e) = 0$ \\
Но $n$ - $\min$ $\Rightarrow q \cdot e \neq 0, q \cdot e = x$.
Итого, $p \cdot x = 0 \Rightarrow p \cdot x \cdot x^{-1} = 0 \Rightarrow p \cdot e = 0$!!!
\end{proof}

\begin{defn}
Поле называется простым, если оно не содержит собственных подполей.
\end{defn}

$]~\mathbb{K}$ - поле, $char(\mathbb{K}) = p$, $p$ - простое.\\
$\exists k \subset \mathbb{K}$ - простое подполе. $k$ - замкнуто относительно сложения, умножение, взятия обратного.\\
$k \cong F_p$\\

$\BB{K}$ - конечное поле, $char(K) = p$\\
$\exists k \cong F_p \subset \BB{K}$\\
Пусть $[\BB{K}:k] = n$. Тогда $|K| = q = p^n$.
Рассмотрим полином $x^q-x$. Покажем, что $x^q - x \equiv 0, \forall x \in \BB{K}$.
\[]x = 0:~~~0^q - 0 = 0\]
\[]x \neq 0:~~~x^{q-1} - 1 \overset{?}{=} 0\]
Рассмотрим $\BB{K}^* = \BB{K} \setminus \{0\}$. $\BB{K}^*$ - группа по умножению. $|\BB{K}^*| = q - 1$. По теореме
(если $|G| = m$, то $g^m = 1$) получаем, что $x^{q - 1} = 1$. \\
Таким образом, $\forall \alpha \in \BB{K}$, $\alpha$ - корень уравнения $x^{p^n} - x$. \\
\[x^{p^n} - x = \underset{\alpha \in \BB{K}}{\prod}(x - \alpha)\]
У многочлена $p^n$ корней. В поле $\BB{K}$ $p^n$ элементов. Можно построить поле $\BB{K}$, найдя все корни этого
уравнения.
