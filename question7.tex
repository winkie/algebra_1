\section{Действие группы на себе сопряжениями. Сопряженные элементы. Классы сопряженности. Формула классов.}

Для всякого $x \in G$ определим отображение $\sigma_x: G \rightarrow G$ формулой $\sigma_x(y) = x^{-1}yx$. Отображение
определяет действие группы на себе, называемое \emph{сопряжением}. В действительности каждое $\sigma_x$ является
автоморфизмом $G$, т.е. для всех $y, z \in G$ имеем:
\begin{center} $\sigma_x(yz) = \sigma_x(y)\sigma_x(z)$\\ \end{center}
и $\sigma_x$ обладает обратным $\sigma_{x^{-1}}$.

Орбиты данного действия суть \emph{классы сопряженности}.

\begin{defn}
Централизатором элемента $g \in G$ называется множество $C_x = \{g_1 \in G | g^{-1}_1gg_1 = g$, т.е. $gg_1 = g_1g\}$
\end{defn}


Видим, что отображение $x \mapsto \sigma_x$ есть гомоморфизм группы $G$ в ее группу автоморфизмов. Ядро этого гомоморфизма
- нормальная подгруппа в $G$, состоящая из всех таких $x \in G$, что $x^-1yx = y$ для каждого $y \in G$, т.е. из
объединения всех централизаторов. 

Отметим, что посредством сопряжений $G$ действует также на множестве своих подмножеств. Действительно, пусть $S$ - множество
всех подмножеств в $G$ и пусть $A \in S$ - одно из них. Тогда $x^{-1}Ax$ тоже подмножество $G$, которое можно обозначить
через $\sigma_x(A)$, и легко проверяется, что $\sigma_x$ определяет действие группы $G$ на $S$. Отметим, кроме того, что
если $A$ - подгруппа $G$, то $x^{-1}Ax$ тоже подгруппа, так что $G$ действует посредством сопряжений и на множестве своих
подгрупп.

\begin{defn}
Пусть $A, B$ - два подмножеста в $G$. Говорим, что они \emph{сопряжены}, если $\exists x \in G: B = x^{-1}Ax$.
\end{defn}

Пусть $x, y$ - элементы группы $G$. Они называются коммутирующими, если $xy = yx$. Множество всех элементов $x \in G$,
коммутирующих со всеми элементами группы $G$, есть подгруппа $G$. Назовем её \emph{центром} группы G. Пусть $G$ действует
на себе посредством сопряжений. Тогда элемент $x$ лежит в центре в том и только в том случае, если орбита этого элемента
совпадает с ним самим и, таким образом, состоит из одного элемента. Вообще, индекс орбиты (класса сопряженности) элемента
$x$ равен индексу его централизатора. Следовательно, если $G$ - конечная группа, то формула орбит принимает вид:
\begin{center} $[G:1] = {\underset{x \in CS(X)}{\sum}}[G : C_x]$,\\
где $CS(X)$ - множество различных представителей всех классов сопряженности \end{center}