\section{Коммутативные кольца. Гомоморфизмы колец. Моно- и эпиморфизмы. Характеризация мономорфизмов.}

\begin{defn}
Кольцом ${A, +, \ast}$ называется множество с 2-мя бин. операциям: $+: A \times A \rightarrow A$ и
$\ast: A \times A \rightarrow A$ и удовлетворяющее следующим условиям:
\begin{enumerate}
 \item $\{A, +\}$ - Абелева группа.
 \begin{enumerate}
  \item $a + b = b + a$
  \item $(a + b) + c = a + (b + c)$
  \item $\exists0: a + 0 = a$
  \item $\forall a \exists -a: a + (-a) = 0$
 \end{enumerate}

 \item $(ab)c = a(bc)$ \\ $\exists e: ea = ae = a$
 \item * $ab = ba$ (коммутативное кольцо)
 \item $a(b+c) = ab + ac$
\end{enumerate}
\end{defn}

Пример: $Z/nZ$ - коммутативное кольцо, $M_n(Z)$ - некоммутативное кольцо.

\begin{defn}
$A, B$ - кольца.\\
$f: A \rightarrow B$ - гомоморфизм колец, если:
\begin{enumerate}
 \item $f(a \ast b) = f(a) \ast f(b)$
 \item $f(a + b) = f(a) + f(b)$
 \item $f(0_A) = 0_B$
 \item $f(1_A) = 1_B$
\end{enumerate}

\begin{flushleft}
Инъективный гомоморфизм - мономорфизм. \\
Сюрьективный гомоморфизм - эпиморфизм. \\
Биективный гомоморфизм - изоморфизм.
\end{flushleft}
\end{defn}

Мы будем рассматривать коммутативные кольца!

\begin{thm}
WARNING! COPYPASTE!
$ \phi $ - мономорфизм $ \Leftrightarrow ker \phi = \{e\} $
\end{thm}
\begin{proof}
  $ [\Rightarrow] $ Пусть $ \exists g \ne e $ $ \phi(g) = e $. Но $ \phi(e) = e $. Таким образом $ g \ne e, \phi(g) = \phi(e) $. 
  Противоречие инъективности. \newline
  $ [\Leftarrow] $ Пусть $ \exists g_{1} \ne g_{2}, \phi(g_{1}) = \phi(g_{2}) $. Тогда $ \phi(g_{1})\phi(g_{2})^{-1} = e $,
  а это значит, что $ g_{1}g_{2}^{-1} \ne e $ и $ g_{1}g_{2}^{-1} \in ker f $. Противоречие тривиальности ядра. 
\end{proof}

\begin{defn}
$f$ - гомоморфизм колец.\\
$Ker(f) = \{a \in A| f(a) = 0_B\}$ - ядро гомоморфизма.
\end{defn}

Свойства ядра:
\begin{enumerate}
 \item $a_1, a_2 \in Ker(f) \Rightarrow a_1 + a_2 \in Ker(f)$
 \item $0_A \in Ker(f)$
 \item $a \in Ker(f), b \in A \Rightarrow ba \in Ker(f)~\&~ab \in Ker(f)$
\end{enumerate}

