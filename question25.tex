\section{Алгоритм Берлекэмпа для разложения полиномов над конечным полем.}

Пусть $f \in k[x], char(k) = p$ и $f = f_1 \cdot f_2 \cdot f_3 \cdots f_m$. Т.е. $f$ свободен от квадратов. \\
Задача стоит в отыскании $f_i$ по заданному $f$.

\textbf{Первый шаг} - решение уравнения $g^p \equiv g \mod f$.\\
Заметим, что т.к. $\frob(g) = g^p$ - линейный оператор, то задача сводится к нахождению матрицы оператора, а затем к
нахождению собственного подпространства оператора. \\
Итак, для нахожденя матрицы оператора необходимо применить фробениус к базисным векторам, т.е. к мономам $1, x, x^2,
\dots, x^{n - 1}$, где $n = deg(f)$.

Например, найдем матрицу фробениуса в поле $F_5[x] / (f = x^4+x^3+2x^2+x+1)$. Базисные вектора - $1, x, x^2, x^3$.
\begin{align}
1^5 = 1 &\equiv 1 \mod f \notag \\
x^5 &\equiv -x^3 + x^2 + 1 \mod f \notag \\
x^{10} = (x^5)^2 &\equiv x^3 + x - 1 \mod f \notag \\
x^{15} = x^5 \cdot x^{10} &\equiv x^3 \mod f \notag 
\end{align}
Таким образом, матрица фробениуса есть:
\[
\begin{array}{l}
~~~~~~~~~1,~~~x,~~x^2,~x^3 \\
F = 
\begin{pmatrix}
1 & 0 & 0 & 0 \\
1 & 0 & 1 & -1 \\
-1 & 1 & 0 & 1 \\
0 & 0 & 0 & 1 \\ 
\end{pmatrix}
\end{array}
\]

Далее, находим собственное подпространство, решая систему 
\[(F - E) \cdot \begin{pmatrix} c_0 \\ c_1\\ \vdots \\ c_n \end{pmatrix} = 0\]

\textbf{Второй шаг} - отщепление множителей. Для этого возьмем вектор $(c_0, c_1, \dots, c_n)$ из собств. подпространства,
составим $g(x) = \underset{i = 0}{\overset{n}\sum}c_ix^i$. Далее, следуя теореме Берлекэмпа будем перебирать:
\begin{align}
\text{НОД}(g - 0&, f) \notag \\
\text{НОД}(g - 1&, f) \notag \\
\text{НОД}(g - 2&, f) \notag \\
&\vdots \notag \\
\text{НОД}(g - (p - 1)&, f), \notag
\end{align}
Если какой-то из делителей нетривиален, то он <<отщепляется>>. И для него рекурсивно запускается алгоритм (очевидно,
процесс заканчивается, если для какого-то множителя нельзя отщепить ни один <<подмножитель>>, т.е. множитель неразложим).\\
Далее из пространства выбирается следующий вектор, линейно-независимый со всеми предыдущими, и процесс повторяется.\\

Таким образом мы отщепим все множители, и, в свою очередь, раздожим их на множители, тем самым разложив исходный многочлен
на неразложимые множители.
