\section{Кольцо полиномов над полем. Кольца главных идеалов. Алгоритм Евклида в кольце полиномов}

\begin{thm}
  $  \mathbb{Z} $ - кольцо главных идеалов. 
\end{thm}

\begin{proof}
  Пусть $ \mathfrak{a} \subset \mathbb{Z} $ - идеал и пусть $ d \in \mathfrak{a}, d > 0 $ \newline 
  $ d = min\{ a \in \mathfrak{a}, a > 0 \} $. Докажем, что $ \mathfrak{a} = (d) $. \newline
  $ [\mathfrak{a} \subset (d)] $ \newline
  Возьмем $ n \in \mathfrak{a} $ и разделим на $ d $, получим \newline $ n = kd + r, \; $ $ 0 \le r < d - 1 $. \newline
  Заметим, что $ n \in \mathfrak{a} $ и $ kd \in \mathfrak{a} $, а значит $ (r = n - kd) \in \mathfrak{a} $, но $ d $
  минимальный элемент принадлежащий идеалу. Таким образом $ r = 0 $ и из этого следует $ n \in (d) $. \newline
  $ [(d) \subset \mathfrak{a}] $ \newline
  Возьмем $ k \in (d) $. То есть $ k = ld $. По определению идеала $ k \in \mathfrak{a} $.
\end{proof}

На самом деле любое \emph{евклидово} кольцо является кольцом главных идеалов. Неформально евклидово 
кольцо это то, в котором существует аналог алгоритма Евклида. Вообще алгоритм Евклида базируется на
фундаментальном свойстве натуральных чисел: \emph{любой отрезок натурального ряда имеет минимальный элемент}.
Так вот более формально кольцо $ R $ называется евклидовым если $ R $ - область целостности и 
$ \exists d : R \rightarrow \mathbb{N} \cup -\infty $ причем $ d(a) = -\infty \Leftrightarrow a = 0 $ и 
возможно деление с остатком, то есть $ \forall a, b \ne 0 \in R $ и имеется представление $ a = bq + r, \;
d(r) < d(b) $. В частности $ K[x] $ - кольцо полиномов над полем
$ K $ - является евклидовым с $ d = deg(f) $ и является кольцом главных идеалов.

\begin{defn}
  Пусть $ \Id{a} = (f_{1}, \cdots, f_{k}) \subset K[x] $ - идеал. \newline
  $ (f_{1}, \cdots, f_{k}) = (g) $. Будем называть g \emph{наибольшим общим делителем многочленов} 
  $ f_{1}, \cdots, f_{k} $ и обозначать $ (f_{1}, \cdots, f_{k}) $ (не путать с идеалом).
\end{defn}

\pagebreak

Алгоритм Евклида позволят найти наибольший общий делитель, получить его линейное представление через
образующие исходного идеала. {\bf Вход}: $ f(x), g(x) \in K[x] $ \newline
{\bf Выход}: $ u(x), v(x) : d(x) = u(x)f(x) + v(x)g(x), \; (d) = (f,g) $

\begin{codebox}
  \Procname{ Алгоритм Евклида}
  \li $ u_{-2} \gets 1, v_{-2} \gets 0 $
  \li $ u_{-1} \gets 0, v_{-1} \gets 1 $
  \li $ p_{0} \gets f, q_{0} \gets g $ 
  \li $ i \gets 0 $
  \li \While $ q_{i} \ne 0 $ 
  \li \Do Разделить $ p_{i} $ на $ q_{i} $ с остатком. 
      \li \Comment Частное $ \phi_{i} $. Остаток $ r_{i} $.
      \li $ p_{i+1} \gets q_{i} $
      \li $ q_{i+1} \gets r_{i} $
      \li \Comment Соотношения Безу
      \li $ u_{i} \gets u_{i-2} - \phi_{i}v_{i-1} $
      \li $ v_{i} \gets v_{i-2} - \phi_{i}u_{i-1} $
      \End
  \li $ d \gets p_{i} $
  \li $ u \gets u_{i}, v \gets v_{i} $
\end{codebox}

ЗАМЕЧАНИЕ: Ясно, что для корректной работы алгоритма перед началом цикла необходимо посчитать $ \phi_{0} $ и 
$ \phi_{1} $.




